\section{History of Speech Synthesis}
\label{history_speech_synthesis}
Speech synthesis can be defined as the artificial generation of speech. Nowadays the process has been facilitated due to the improvements made during the last 70 years in computer technology, making the computer-based speech synthesis systems lead the way supported by their flexibility and their easier access compared to mechanical systems. However, after the first resonators built by Kratzenstein, the fist speaking machine was built and presented to the world in 1791, and was obviously mechanic.

\subsection{Acoustical-Mechanical Speech Machines}
\label{history_mechanical_machines}
The speech machine developed by von Kempelen incorporated models of the lips and the tong, enabling it to produce some consonants as well as vowels. Although Kratzenstein presented his resonators before von Kempelen his speech machine, von Kempelen started his work quite before, publishing a book where he described the studies made on human speech production and the experiments he made with his speech machine over 20 years of work \cite{vonKempelen}.  

The machine was composed by a pressure chamber, acting as lungs, a vibrating reeds in charge of the functions of the vocal cords and a leather tube that was manually manipulated in order to change its shape as the vocal tract does in an actual person, producing different vowel sounds. It had four separate constricted passages, controlled by the fingers, to generate consonants. Von Kempelen also included in his machine a model of the vocal tract with a hinged tongue and movable lips so as to create plosive sounds \cite{Schroeder93, LemmettyMSc, flanagan_1973_speech}. 

\begin{figure}[htb]
	\begin{center}
		\includegraphics[width=\textwidth]{images/von_kempelen_machine.jpg}
		\caption{Reconstruction of von Kempelen's speech machine made by Wheatstone \cite{flanagan_book}}
		\label{fig:speech_machine}
	\end{center}
\end{figure}

Inspired by von Kempelen, Charles Wheatstone built an improved version of the speech machine, capable of producing vowels, consonants, some combinations and even some words. In Figure \ref{fig:speech_machine} a scheme of the machine constructed by Wheatstone is presented. Alexader Graham Bell saw the reconstruction built by Wheatstone at an exposition and, encouraged and helped by his father, made his own speaking machine, starting his way towards the contribution in the invention of the telephone.

The research with mechanical items modeling the vocal system did not give any significant improvement during the following decades, leaving the door open to alternative systems to take the lead: the electrical synthesizers with a major breakthrough: the vocoder.

\subsection{Electrical Synthesizers: The Vocoder}
\label{history_vocoder}
The first electrical device was presented to the world by Stewart in 1922 \cite{Klatt87}. It consisted on a buzzer acting as the excitation followed by two resonant circuits modeling the vocal tract. The device was able to create single static vowel sounds with two lowest formants but not any consonant nor connected sounds. A similar type of synthesizer was built by Wagner \cite{flanagan_book}, consisting on four parallel electrical resonators and excited by a buzz, capable of generating the vowel spectra when the proper combination of the outputs of the four resonators was made.

In New York's World fair 1939 \cite{flanagan_book, Klatt87, flanagan_1973_speech}, Homer Dudley presented what was consider the first full electrical synthesis device: the VODER. It was inspired by the vocoder developed at Bell Laboratoies some years earlier, which analyzed the speech into slowly varying acoustics parameters that drove the synthesizer to produce a an approximation of the speech signal. The VODER consisted of wrist bar for selecting a voicing or noise source and a foot pedal to control the fundamental frequency. The source signal was routed through ten band-pass filters controlling their output levels with the fingers \cite{LemmettyMSc}. In Figure \ref{fig:voder} the VODER structure is graphically described. As you can imagine, it was not an easy task to synthesize a sentence on this device and the speech quality and intelligibility were far from acceptable, but he demonstrated the potential to produce synthetic speech.

\begin{figure}[htb]
	\begin{center}
		\includegraphics[width=\textwidth]{images/voder.jpg}
		\caption{VODER synthesizer \cite{Klatt87}}
		\label{fig:voder}
	\end{center}
\end{figure}

The demonstration of the VODER stimulated the scientific community and more people become interested in artificial speech generation. In 1951, Franklin Cooper lead the development of a Pattern Playback synthesizer \cite{Klatt87, flanagan_1973_speech}. The device developed at the Haskins Laboratories used optically recorded spectrogram patterns on a transparent belt to regenerate the audio signal. 

Walter Lawrence introduced in 1953 his Parametric Artificial Talker (PAT), the first formant synthesizer \cite{Klatt87}. It consisted of three parallel electronic resonators excited by a buzz or noise and a moving glass slide converted painted patterns into six different time functions to control the three formant frequencies, voicing amplitude, noise amplitude and the fundamental frequency. 