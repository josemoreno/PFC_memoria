
\documentclass[english,12pt,a4paper,pdftex]{article}
%% This package is required
%% Choose your school from arts, biz, chem, elec, eng, sci.
%% Choose the character encoding scheme used by your editor: utf8, latin1

\usepackage[elec,utf8]{aaltothesis} % this is the default in aaltothesis.sty
%% Use this if you run pdflatex and use jpg/pdf-format pictures.

\usepackage{graphicx}

%% Use the macros in this package to change how the hyperref package below 
%% typesets its hypertext -- hyperlink colour, font, etc. See the package
%% documentation. It also defines the \url macro, so use the package when 
%% not using the hyperref package.
%\usepackage{url}

%% Use this if you want to get links and nice output with pdflatex
\usepackage[pdfpagemode=None,colorlinks=true,urlcolor=red,%
linkcolor=blue,citecolor=black,pdfstartview=FitH]{hyperref}

%% Use this if you write hard core mathematics, these are usually needed
\usepackage{amsfonts,amssymb,amsbsy}  

%% Horizontal margins, DO NOT TOUCH!
\setlength{\hoffset}{-1in}
\setlength{\oddsidemargin}{35mm}
\setlength{\evensidemargin}{25mm}
\setlength{\textwidth}{15cm}
%% Vertical margins, DO NOT TOUCH!
\setlength{\voffset}{-1in}
\setlength{\headsep}{7mm}
\setlength{\headheight}{1em}
\setlength{\topmargin}{25mm-\headheight-\headsep}
\setlength{\textheight}{23cm}

%% Output starts here
\begin{document}
%% Change the school field to describe your school if the autimatically 
%% set name is wrong
% \university{aalto University}{aalto-Yliopisto}
% \school{School of Electrical Engineering}{SähköTekniikan korkeakoulu}

%% Vain kandityölle: Korjaa seuraavat vastaamaan koulutusohjelmaasi
%%
%% Only for B.Sc. thesis: Choose your degree programme. 
%\degreeprogram{Electronics and electrical engineering}%

%% Only for M.Sc. and Licentiate thesis: Choose your department,
%% professorship and professorship code. 
\department{Department of Signal Processing and Acoustics}%
{Signaalinkäsittelyn ja akustiikan laitos}
\professorship{Speech and Language Processing}{}
\code{S-89}

%% Choose one of these:
%\univdegree{BSc}
%\univdegree{MSc}
%% Added by Jose: For the non-finnish/swedish, the Lic degree is half of a PhD.
%\univdegree{Lic}
%% Add by Jose: For engineering exchange students (old plan, not Bolonia), PFC in Spain
%% IMPORTANT: FPr is only valid with English!!!
\univdegree{FPr}
%% Should be self explanatory...
\author{Jose Mariano Moreno Pimentel}

%% Your thesis title. If the title is very long and the latex 
%% does unsatisfactory job of breaking the lines, you will have to
%% break the lines yourself with \\ control character. 
%% Do not hyphenate titles.
\spanishtitle{Efectos del Ruido en un Sistema Estad\'istico de S\'intesis de Voz}
\thesistitle{Effects of Noise on a Speaker-Adaptive Statistical Speech Synthesis}{}

\place{Espoo}
%% For B.Sc. thesis use the date when you present your thesis. 
\date{31.03.2014}

%% B.Sc. or M.Sc. thesis supervisor 
%% Note the "\" after the comma. This forces the following space to be 
%% a normal interword space, not the space that starts a new sentence. 
\supervisor{Prof.\ Mikko Kurimo}{Prof.\ Mikko Kurimo}

%% B.Sc. or M.Sc. thesis advisors(s). 
%%
%% Note that there has been a change in the official EN translation
%% of the Finnish title ``ohjaaja'' which in the previous version (1.5) 
%% of this document was called ``instructor''. The recommended
%% translation is now ``advisor''.  
%% However, the LaTeX internal variable remains \instructor
%% as there is little point to change the variable name. 
%%
%\instructor{Prof. Pirjo Professori}{Prof. Pirjo Professori}
%\instructor{D.Sc.\ (Tech.) Olli Ohjaaja}{TkT Olli Ohjaaja}
\instructor{M.Sc.\ (Tech.) Reima Karhila}{DI Reima Karhila}

%% Aalto logo: syntax:
% \uselogo{aaltoRed|aaltoBlue|aaltoYellow|aaltoGray|aaltoGrayScale}{?|!|''}
%% Logo language is set to be the same as the document language.
\uselogo{aaltoBlue}{''}

%% Create the coverpage
\makecoverpage

%% Force new page so that English abstract starts from a new page
\newpage
%
%\begin{resumencastellano}[english]
En este proyecto se estudian los efectos de distintos tipos de ruido sobre un sistema estad\'istico de s\'intesis de voz adaptativo al locutor.

Los s\'istemas estad\'isticos de s\'intesis de voz se han vuelto bastante populares en los \'ultimos a\~nos gracias a su flexibilidad y bajos requ\'isitos de memoria en comparaci\'on con otros s\'istemas tales como los concatena
\end{resumencastellano}
%% English abstract, uncomment if you need one. 
%% 
%% Abstract keywords
\keywords{speech synthesis, synthetic speech, TTS, HMM, noise robust, TTS adaptation}
%% Abstract text
\begin{abstractpage}[english]
 Your abstract in English. Try to keep the abstract short, approximately 
 100 words should be enough. Abstract explains your research topic, 
 the methods you have used, and the results you obtained.  
\end{abstractpage}
%% Note that 
%% if you are writting your master's thesis in English place the English
%% abstract first followed by the possible Finnish abstract

%% Preface
%\mysection{Esipuhe}
\mysection{Acknowledgments}

\vspace{5cm}
Otaniemi, 24.9.2013

\vspace{5mm}
{\hfill Jose M.\ Moreno \hspace{1cm}}

%% Force new page after preface
\newpage

%% Table of contents. 
\thesistableofcontents

\newpage
%% List of figures
\listoffigures

\newpage
%% List of tables
\listoftables

%% Symbols and abbreviations
\mysection{Symbols and abbreviations}
\subsection*{Symbols}
\begin{tabular}{l l}
	$\lambda$		& Hidden Markov model\\
	$F_{0}$		& Fundamental frequency\\
	$\mathbf{O}$	& Observation sequence vector\\
	$P$			 	& Probability\\
	$\mathbf{Q}$	& State sequence vector
\end{tabular}
\subsection*{Opetators}

\subsection*{Abbreviations}
\begin{tabular}{l l}
	CMLLR		& Constrained Maximum-Likelihood Linear Regression\\
	CSMAPLR		& Constrained Structural Maximum A Posteriori Linear Regression\\
	EM 			& Expectation-Maximization\\
	HMM			& Hidden Markov Model\\
	LP			& Linear Prediction\\
	LPC 		& Linear Predictive Coding\\
	LSP			& Line Spectral Pair\\
	MAP			& Maximum A Posteriori\\
	SAT			& Speaker-Adaptive Training\\
	SMAP 		& Structural Maximum A Posteriori\\
	STRAIGHT	& Speech Transformation and Representation using Adaptive \\
				& Interpolation of weiGHT spectrum\\
	TTS			& Text-To-Speech
\end{tabular}
%% Corrects the page numbering, there is no need to change these
\cleardoublepage
\storeinipagenumber
\pagenumbering{arabic}
\setcounter{page}{1}

\section{Introduction}
\label{intro}
\thispagestyle{empty}

Speech synthesis is not a recent ambition in mankind history. The earliest attempts to synthesize speech are only legends starring Gerbert d'Aurillac (died 1003 A.D.), also known as Pope Sylvester II. The pretended system used by him was a brazen head: a legendary automaton imitating the anatomy of a human head and capable to answer any question. Back in those days, the brazen heads were said to be owned by wizards. Following Pope Sylvester II, some important characters in mankind history  were reputed to have one of these heads, such as Albertus Magnus or Roger Bacon.

During the 18th century, Christian Kratzenstein, a German-born doctor, physicist and engineer working at the Russian Academy of Sciences, was able to built acoustics resonators similar to the human vocal tract. He activated the resonators with vibrating reeds producing the the five long vowels: /a/, /e/, /i/, /o/ and /u/.

Almost at the end of the 18th century, in 1791, Wolfgang von Kempelen presented his Acoustic-Mechanical Speech Machine \cite{vonKempelen}, which was able to produce single sounds and some combinations. During the first half of the 19th century, Charles Wheatstone built his improved and more complicated version of Kempelen's Acoustic-Mechanical Speech Machine, capable of producing vowels, almost all the consonants, sound combinations and even some words.	

In the late 1800's, Alexander Graham Bell also built a speaking machine and did some questionable experiments changing with his hands the vocal tract of his dog and making the dog bark in order to produce speech-like sounds \cite{Schroeder93, LemmettyMSc}.

Before World War II, Bell labs developed the vocoder, which analyzed and extract fundamentals tone and frequency from speech. In the 1950's, the first computer based speech synthesis systems were created and in 1968 the first general English text-to-speech (TTS) system was developed at the Electrotechnical Laboratory, Japan \cite{Klatt87}. From that time on, the main branch of speech synthesis development has being focused on electronic systems, but research conducted on mechanical synthesizers has not been abandoned \cite{mechSynthWeb, mechSynth}.

All the different kind of systems described pursued the same goal: produce natural sounding speech, which is the main goal of speech synthesis. As an extra requirement to this main goal, TTS systems aim to create the speech from arbitrary texts given as inputs, increasing the difficulty. It is easy to assume that a considerably amount of data is needed in order to cover all the possible sounds combinations in a given text. Moreover, the current trend in TTS aims towards generating different speaking styles with different speaker characteristics and emotions expressed with our voice, enlarging the spectrum of the characteristics of the voice to take into account and its differences depending on the context, increasing the amount of data needed to develop the final system. 

It must be pointed out that among all the different techniques used nowadays to synthesize speech, someones are not focused in maximum naturalness but intelligibility or high-speed synthesized speech. Although naturalness still a main issue, the final target, e.g. helping impaired people to navigate computers using a screen reader, forces to prioritize some other characteristics before naturalness. 

Among the synthesis techniques, when talking about fulfilling the general requirements presented so far: naturalness, speaker characteristics, emotions, style, etc., unit selection technique and Hidden Markov Model (HMM) approach stand out. Although unit selection synthesis provides the greatest naturalness, it does not allow an easy adaptation of a TTS system to other speakers or speaking styles, requiring a large amount of data due to the selection and concatenation used in this kind of synthesis, making this technique not suitable for example to embedded systems. On the other hand, HMM-based systems make easier to use adaptation techniques and require less memory, meaning a reduction in naturalness.

Statistical parametric, or HMM, speech synthesis is very popular nowadays thanks to the advantages previously commented. We can find various vocoders currently being used in HMM-based systems, but the Speech Transformation and Representation using Adaptive Interpolation of weiGHT spectrum (STRAIGHT) vocoder is the most commonly used and the most established one. However, due to the degradation in naturalness suffered in HMM-based systems, a new vocoder is being developed trying to solve this issue: the GlottHMM vocoder, which estimates the real glottal signal and the real vocal tract associated to it, thus producing a more natural voice. 

So far memory requirements and the amount of data needed to build the system have been pointed as some of the weak points in speech synthesis systems. Particularly, collecting data is not an easy task since speech synthesis systems need high quality recordings covering different contexts. Moreover, when using speaker adaptive systems certain amount of audio recordings will be needed from a substantial number of speakers. Adapting an average voice model, made out from high quality recorded audio of different speakers, with non high quality recordings would facilitate the access to a bigger number of target voices.

Noisy conditions were explored in speech recognition systems before being tested in synthesis system. Speech recognition is obviously highly related to speech synthesis. The analysis done to the audio recordings is the same in both cases, thus the same concepts used in recognition can be applied to speech synthesis systems. Nevertheless, speech recognition techniques under noisy conditions cannot satisfy all the synthesis nee

In this project the possibility of synthesizing speech from noisy data will be explored. This has been explored before, using STRAIGHT vocoder and obtaining 
\section{History of Speech Synthesis}
\label{history_speech_synthesis}
Speech synthesis can be defined as the artificial generation of speech. Nowadays the process has been facilitated due to the improvements made during the last 70 years in computer technology, making the computer-based speech synthesis systems lead the way supported by their flexibility and their easier access compared to mechanical systems. However, after the first resonators built by Kratzenstein, the fist speaking machine was built and presented to the world in 1791, and was obviously mechanic.

\subsection{Acoustical-Mechanical Speech Machines}
\label{history_mechanical_machines}
The speech machine developed by von Kempelen incorporated models of the lips and the tong, enabling it to produce some consonants as well as vowels. Although Kratzenstein presented his resonators before von Kempelen presented his speech machine, von Kempelen started his work quite before, publishing a book where he described the studies made on human speech production and the experiments he made with his speech machine over 20 years of work \cite{vonKempelen}.  

The machine was composed by a pressure chamber, acting as lungs, a vibrating reeds in charge of the functions of the vocal cords and a leather tube that was manually manipulated in order to change its shape as the vocal tract does in an actual person, producing different vowel sounds. It had four separate constricted passages, controlled by the fingers, to generate consonants. Von Kempelen also included in his machine a model of the vocal tract with a hinged tongue and movable lips so as to create plosive sounds \cite{Schroeder93, LemmettyMSc, flanagan_1973_speech}. 

\begin{figure}[htb]
	\begin{center}
		\includegraphics[width=\textwidth]{images/von_kempelen_machine.jpg}
		\caption{Reconstruction of von Kempelen's speech machine made by Wheatstone \cite{flanagan_book}}
		\label{fig:speech_machine}
	\end{center}
\end{figure}

Inspired by von Kempelen, Charles Wheatstone built an improved version of the speech machine, capable of producing vowels, consonants, some combinations and even some words. In Figure \ref{fig:speech_machine} a scheme of the machine constructed by Wheatstone is presented. Alexader Graham Bell saw the reconstruction built by Wheatstone at an exposition and, encouraged and helped by his father, made his own speaking machine, starting his way towards the contribution in the invention of the telephone.

The research with mechanical items modelling the vocal system did not give any significant improvement during the following decades, leaving the door open to alternative systems to take the lead: the electrical synthesizers with a major breakthrough: the vocoder.

\subsection{Electrical Synthesizers: The Vocoder}
\label{history_vocoder}
The first electrical device was presented to the world by Stewart in 1922 \cite{Klatt87}. It consisted of a buzzer acting as the excitation followed by two resonant circuits modelling the vocal tract. The device was able to create single static vowel sounds with two lowest formants but not any consonant nor connected sounds. A similar type of synthesizer was built by Wagner \cite{flanagan_book}, consisting on four parallel electrical resonators and excited by a buzz, capable of generating the vowel spectra when the proper combination of the outputs of the four resonators was made.

In New York's World fair 1939 \cite{flanagan_book, Klatt87, flanagan_1973_speech}, Homer Dudley presented what was consider the first full electrical synthesis device: the VODER. It was inspired by the vocoder developed at Bell Laboratoies some years earlier, which analyzed the speech into slowly varying acoustics parameters that drove the synthesizer to produce a an approximation of the speech signal. The VODER consisted of wrist bar for selecting a voicing or noise source and a foot pedal to control the fundamental frequency. The source signal was routed through ten band-pass filters controlling their output levels with the fingers \cite{LemmettyMSc}. In Figure \ref{fig:voder} the VODER structure is graphically described. As you can imagine, it was not an easy task to synthesize a sentence on this device and the speech quality and intelligibility were far from acceptable, but he demonstrated the potential to produce synthetic speech.

\begin{figure}[htb]
	\begin{center}
		\includegraphics[width=\textwidth]{images/voder.jpg}
		\caption{VODER synthesizer \cite{Klatt87}}
		\label{fig:voder}
	\end{center}
\end{figure}

The demonstration of the VODER stimulated the scientific community and more people become interested in artificial speech generation. In 1951, Franklin Cooper lead the development of a Pattern Playback synthesizer \cite{Klatt87, flanagan_1973_speech}. The device developed at the Haskins Laboratories used optically recorded spectrogram patterns on a transparent belt to regenerate the audio signal. 

Walter Lawrence introduced in 1953 his Parametric Artificial Talker (PAT), the first formant synthesizer \cite{Klatt87}. It consisted of three parallel electronic resonators excited by a buzz or noise and a moving glass slide converted painted patterns into six different time functions to control the three formant frequencies, voicing amplitude, noise amplitude and the fundamental frequency. 

Simultaneously, the OVE I was introduced as the first cascade formant synthesizer. As its name suggest, the resonators in the OVE I were connected in cascade. A new version of this synthesizer was aired ten years later. The OVE II consisted on separate parts modelling the vocal tract to differentiate between vowels, nasals and obstruent consonants. It was excited by voicing, aspiration noise and fricative noise.

PAT and OVE developers engaged in a discussion about whether the transfer function of the acoustic tube should be modelled in parallel or in cascade. After a few years studying both systems, John Holmes presented his parallel formant synthesizer \cite{Klatt87}, obtaining a good quality in the synthesized voice.

Linear Predictive Coding (LPC) was first used in some experiments in the mid 1960's \cite{Schroeder93} and it was used in low-cost systems in 1980. The method was modified and nowadays is very useful and it can be found in many systems. 

Different TTS systems appeared during the following years. Probably, the most remarkable one was the system developed by Dennis Klatt, the Klattalk, using a new sophisticated voicing source \cite{Klatt87}, forming along MITalk, developed at the M.I.T., the basis for many systems that came after them and also many ones used nowadays \cite{LemmettyMSc}.

The modern technology used in speech synthesis involve quite sophisticated algorithms. As said in Section \ref{intro}, HMM-based systems are very popular. Actually, HMMs have been used in speech recognition for more than 30 years. In Section \ref{hmm_synthesis} a detailed description of these systems is given, as is the technique used in this project.

HMM-based systems need to extract some features or parameters from the voice, and at that point is where the vocoder comes into action. Originally, the vocoder was developed to compress the speech in telecommunication systems in order to save bandwidth by transmitting the parameters of a model in stead of the speech, as they change quite slowly compared to a speech waveform. Despite its original objective, vocoders are the interface between the audio and the speech synthesis systems, extracting the features needed to model the system and synthesizing speech from the features generated by the system. In this project we will compare two vocoders, STRAIGHT and GlottHMM. They are both described in Section \ref{vocoders}. 

\section{Speech Synthesis Systems}
\label{speec_synthesis_systems}
In this project we will use a HMM-based TTS system, but there are many different speech synthesis systems with their own advantages an disadvantages. In this section we will introduce the general architecture of a TTS system and diverse synthesis methods.

\subsection{TTS Architecture}
The main goal of a TTS system is to synthesize utterances from an arbitrary text. It is easy to notice that synthesizing from a text gives an extra flexibility to a synthesis system but also an extra work has to be done to transform that text into the phonetic units required as inputs by the synthesizer. A general diagram of a TTS system is shown in Figure \ref{fig:tts_architecture}.

\begin{figure}[htb]
	\begin{center}
	\includegraphics[width=\textwidth]{images/tts_architecture.jpg}
	\caption{General block diagram of a TTS system \cite{TuomoMSc}}
	\label{fig:tts_architecture}
	\end{center}
\end{figure}

The block representing the text and linguistic analysis is what differences a TTS system from other speech synthesis systems. The analysis made to the text has to generate the phonetic representation needed by the next component and predicting the desired prosody. A larger set of goals will imply a more complex text and linguistic analysis. For example, trying to imitate the speaking style used in by sport broadcaster will need an extra function aiming to figure out the style of the input text. 

The main path followed by the text analysis includes a mandatory text normalization module. It is very important to normalize the text before trying to obtain its phonetic representation, to transform numbers, dates, acronyms and all the particularities that a language admit into a standardized form accepted by the system. Also, this module is in charged of defining how similar spelled words are pronounced, e.g. the verb read has to different pronunciations whether is in the present tense or in the past tense. As it can be seen, text normalization is a complex problem that many researchers are looking for a solution to. An interesting approach is discussed in \cite{Sproat2001}.

Once the text is normalized, i.e. converted to plain letters, the structural properties of the text are analyzed and it is converted to a phonetic level. This last conversion is called the letter-to-sound conversion \cite{Pickett1999}. 

When the input text has gone through the first block represented in Figure \ref{fig:tts_architecture}, the low-level block generates predicts, based on the structural information and the prosodic analysis and tipically using statistical models, the fundamental frequency contour and phone durations. Finally, the speech waveform is generated by the vocoder. 

\subsection{Speech Synthesis Methods}
\label{speech_synthesis_systems_methods}
The generation of the waveform can be carried out in several ways, thus, we can talk about different speech synthesis methods. As written in \cite{TuomoMSc}, the different methods can be divided in two categories attending to whether the speech is generated from parameter, i.e. completely artificial, or real speech samples are used during the process. From all the methods explained in this section, only concatenative synthesis uses real samples to synthesize speech.

\subsubsection{Formant Synthesis}
\label{formant_speech_synthesis}
Formant synthesis is the most basic acoustic speech synthesis method. Based on the source-filter theory, which states that the speech signal can be represented in terms of source and filter characteristics \cite{Fant1970}, models the vocal tract with individually adjustable formant filters. The filters can be connected in serial, parallel or both. The different phonemes are generated by adjusting the center frequency, gain and bandwidth of each filter. Depending on the time intervals taken to do the adjustment, continuous speech can be generated. The source is modelled with voice pulses or noise.

Dennis Klatt's publication of the Klattalk synthesizer (see Section \ref{history_vocoder}) was the biggest boost received by formant synthesis. However, nowadays the quality given by this kind of synthesizers is lower than other newer methods, such as concatenative systems. Even so, formant synthesis is used in many applications such as reading machines for blind people, thanks to its intelligibility \cite{Pickett1999}.

\subsubsection{Articulatory Synthesis}
\label{articulatory_speech_synthesis}
The aim of articulatory synthesis is to model the nature speech in the most possible accurate way. Therefore, this is theoretically the best method in order to achieve high-quality synthetic voices. However, modelling the speech as accurately as possible raises the difficulty. The main setbacks are the difficult implementation needed in an articulatory speech synthesis system and the computational load, limiting this technique nowadays. Despite its currently limitations, articulatory models are being steadily developed and computational resources are still increasing, revealing a promising future.

\subsubsection{Concatenative Synthesis}
\label{concatenative_speech_synthesis}
Concatenative method use prerecorded samples of real speech to generate the synthetic speech. It is easy to deduce that concatenative synthesis stands out from other methods of synthesis in terms of naturalness. There are several unit lengths, such as word, syllable, phoneme, diphone, etc, that are smoothly combined to obtain the speech according to the input text. 

The main problem when using concatenative synthesis are the memory requirements. It is almost impossible to store all the necessary data for various speakers and contexts, making this technique the best one to imitate one specific speaker with one voice quality, but also makes it less flexible. It is difficult to implement adaptation techniques to obtain a different speaking style or a different speaker in concatenative speech. Apart from the storage problem, that thanks to the decrease of computer storage and database techniques is becoming less serious, the discontinues found in the joining points may cause some distortion even though the use of smoothing algorithms.

Concatenative systems may be the most widely used nowadays, but due to the limitations before commented, aboce all the flexibility problem, they might not be the best solutions.

\subsubsection{LPC-Based Synthesis}
\label{lpc_based_speech_synthesis}
As in formant synthesis, in LPC-based synthesis utilizes source-filter theory of speech production. However, in this case the filter coefficients are estimated automatically from a short frame of speech, while in formant synthesis the different parameters are found for individual formant filters. Depending on the segment to be synthesized, the excitation needed is either a periodic signal, when synthesizing voiced segments, or noise, in case the segment is unvoiced. 

Linear Prediction (LP) has being applied in many different fields for a long time and was first used in speech analysis and synthesis in 1967. The idea is to predict a sample data by a linear combination of the previous samples. However, LPC targets not to predict any samples, but to represent the spectral envelope of the speech signal. 

Though the quality of basic LPC vocoder is consider poor, using more sophisticated LPC-based methods can produce high quality synthetic speech. The type of excitation is very important in LPC-based systems \cite{TuomoMSc}, but in its accuracy estimating the speech parameters and a relative computational speed lays the strength of LPC-based synthesis.

\subsubsection{HMM-Based Synthesis}
\label{hmm_based_speech_synthesis}
The use of HMMs in speech synthesis is one widely applied method. It is an statistical model used for modelling the speech parameters extracted from a speech database. Once the statistical models are built, they can be use to generate parameters according a text input that will be use for synthesizing. 

HMM-based synthesizers are able to produce different speaking styles, different speakers and even emotional speech. Other benefits are a smaller memory requirement and better adaptability. This last benefit is very interesting to us. While working with noisy data, the less data corrupted by noise used in the system will probably improve the final results. Thus, constructing a high-quality average model and then taking profit of the adaptability of these systems to use the noisy data to adapt the model, because the adaptation data is always much lower than the training data, seems the correct approach.

On the other hand, naturalness is usually lower in HMM-based systems than in other ones, but it must be said that these systems are usually developing very fast. 

As in this project we will be using HMM-based TTS systems, they are going to be described with more detail in Section \ref{hmm_synthesis}.

\section{HMM-Based Speech Synthesis}
\label{hmm_synthesis}
Statistical parametric speech synthesis has grown in the last decade thanks to the advantages commented in Section \ref{hmm_based_speech_synthesis}: adaptability and memory requirements. In this section HMM-Based Speech Synthesis is explained.

\subsection{Hidden Markov Models}
\label{hmm_syntheis_markov}
As the name suggests, HMM-Based systems are assumed to be a Markov process with unobserved states. 
\section{Vocoders}
\label{vocoders}
So far, 
%\section{Introduction}

%% Leave first page empty
%\thispagestyle{empty}

%% In a thesis, every section starts a new page, hence \clearpage
%\clearpage



%% Three levels of hierarchy in sectioning should be enough

\clearpage

%% The \phantomsection command is nessesary for hyperref to jump to the 
%% correct page, in other words it puts a hyper marker on the page.

\phantomsection
%\addcontentsline{toc}{section}{Viitteet}
\addcontentsline{toc}{section}{References}
\bibliographystyle{ieeetr}
\bibliography{sections/references}



\appendix 
\clearpage
%% Adds the word "Appendices" to the table of contents
\addtocontents{toc}{\protect\contentsline{section}{Appendices}{}{appendix}}

 %% appendix example (starts with section) in Finnish

%% Equations, tables and figures have their own numbering in Appendices, REMEMBER TO DO EVERY TIME YOU START AN APPENDIX, THE LETTER A IS THE APPENDIX INDEX
\renewcommand{\theequation}{A\arabic{equation}}
\setcounter{equation}{0}  
\renewcommand{\thefigure}{A\arabic{figure}}
\setcounter{figure}{0}
\renewcommand{\thetable}{A\arabic{table}}
\setcounter{table}{0}


\end{document}
